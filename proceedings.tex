\documentclass{sigchi}

% Use this command to override the default ACM copyright statement
% (e.g. for preprints).  Consult the conference website for the
% camera-ready copyright statement.

%% EXAMPLE BEGIN -- HOW TO OVERRIDE THE DEFAULT COPYRIGHT STRIP -- (July 22, 2013 - Paul Baumann)
% \toappear{Permission to make digital or hard copies of all or part of this work for personal or classroom use is      granted without fee provided that copies are not made or distributed for profit or commercial advantage and that copies bear this notice and the full citation on the first page. Copyrights for components of this work owned by others than ACM must be honored. Abstracting with credit is permitted. To copy otherwise, or republish, to post on servers or to redistribute to lists, requires prior specific permission and/or a fee. Request permissions from permissions@acm.org. \\
% {\emph{CHI'14}}, April 26--May 1, 2014, Toronto, Canada. \\
% Copyright \copyright~2014 ACM ISBN/14/04...\$15.00. \\
% DOI string from ACM form confirmation}
%% EXAMPLE END -- HOW TO OVERRIDE THE DEFAULT COPYRIGHT STRIP -- (July 22, 2013 - Paul Baumann)

% Arabic page numbers for submission.  Remove this line to eliminate
% page numbers for the camera ready copy 
% \pagenumbering{arabic}

% Load basic packages
\usepackage{balance}  % to better equalize the last page
\usepackage{graphics} % for EPS, load graphicx instead 
%\usepackage[T1]{fontenc}
\usepackage{txfonts}
\usepackage{times}    % comment if you want LaTeX's default font
\usepackage[pdftex]{hyperref}
% \usepackage{url}      % llt: nicely formatted URLs
\usepackage{color}
\usepackage{textcomp}
\usepackage{booktabs}
\usepackage{ccicons}
\usepackage{todonotes}

\usepackage[utf8]{inputenc}

% llt: Define a global style for URLs, rather that the default one
\makeatletter
\def\url@leostyle{%
  \@ifundefined{selectfont}{\def\UrlFont{\sf}}{\def\UrlFont{\small\bf\ttfamily}}}
\makeatother
\urlstyle{leo}

% To make various LaTeX processors do the right thing with page size.
\def\pprw{8.5in}
\def\pprh{11in}
\special{papersize=\pprw,\pprh}
\setlength{\paperwidth}{\pprw}
\setlength{\paperheight}{\pprh}
\setlength{\pdfpagewidth}{\pprw}
\setlength{\pdfpageheight}{\pprh}

% Make sure hyperref comes last of your loaded packages, to give it a
% fighting chance of not being over-written, since its job is to
% redefine many LaTeX commands.
\definecolor{linkColor}{RGB}{6,125,233}
\hypersetup{%
  pdftitle={SIGCHI Conference Proceedings Format},
  pdfauthor={LaTeX},
  pdfkeywords={SIGCHI, proceedings, archival format},
  bookmarksnumbered,
  pdfstartview={FitH},
  colorlinks,
  citecolor=black,
  filecolor=black,
  linkcolor=black,
  urlcolor=linkColor,
  breaklinks=true,
}

% create a shortcut to typeset table headings
% \newcommand\tabhead[1]{\small\textbf{#1}}

% End of preamble. Here it comes the document.
\begin{document}

\title{Integration of Coding and Social Features to Increase Success on FLOSS Projects}

\numberofauthors{3}
\author{%
  \alignauthor{Felipe T. Gomes\\
    \affaddr{University of São Paulo}\\
    \affaddr{ICMC - USP}\\
    \affaddr{São Carlos, Brazil}\\
    \email{felipe.tassario.gomes@usp.br}}\\
  \alignauthor{Sérgio O. Campos\\
    \affaddr{University of São Paulo}\\
    \affaddr{ICMC - USP}\\
    \affaddr{São Carlos, Brazil}\\
    \email{seocam@usp.br}}\\
}

\maketitle

\begin{abstract}
  UPDATED---\today. TODO (Abstracts should be about 150 words and are required)
\end{abstract}

\category{H.5.m.}{Information Interfaces and Presentation 
  (e.g. HCI)}{Miscellaneous} \category{See
  \url{http://acm.org/about/class/1998/} for the full list of ACM
  classifiers. This section is required.}{}{}

\keywords{distributed software development; social coding; information system success. }

\section{Introduction}

In this article we analyze a real-world Computer-Supported Collaborative Work (CSCW) system in order to understand the trade-offs of integrating social features in a distributed development environment of Free and Open-Source Software (FLOSS).

The field of CSCW is a relatively new area of research in the greater area of computer science research. From its beginning, in the late 80s, the field has produced both academic research into the fundamental concepts of the area, as well as tools derived from this research, specifically designed to support collaborative work.

With the advance of the availability, ubiquity and reach of the Internet in the last 15 years, vast opportunities to use on-line tools to perform collaborative have appeared, and the number of tools created to support distributed, synchronous or asynchronous work, through the Internet, has seen a sharp rise. There is even a strong presence of tools being developed in Brazil for these purposes, adapted to the local and national needs and requirements.

One of theses systems is the new portal for Public Free Software ("Software Público Brasileiro – SPB"), being developed by the University of Brasília (UnB), which goal is to support the activities of software development in the public sector and encourage the collaboration between communities and individuals.

The new version of the Portal (nSPB) is currently in final tests phase (beta) and when released it will replace the old version (oSPB), on-line since 2007. The nSPB was completely redesigned with focusing on the integration of coding and social features. This paper compares both versions to better understand the benefits and drawbacks of this integration and measure Information System Success \cite{Crowston2006}.

\section{The Relation between Social Interaction and FLOSS Success}

As any software FLOSS projects can be developed and a single individual or in a closed group (as proprietary software) but most of it is actually developed by organizationally- and geographically-distributed developers \cite{Crowston2012}, and thus it requires level of social interaction.

While social interaction can help teams to achieve higher level of awareness improving teamwork [CITATION NEEDED:AULA 3 CSCW] its also important to consider that this interaction can diverge developers from taskwork [CITATION NEEDED:AULA 3 CSCW].

To understand both taskwork and teamwork we use metrics from information system success, one of the most widely used dependent variables in information system (IS) research \cite{Crowston2006}, adapted to Open-Source by Crowston in 2006.

TODO: Introduce: Development of Stable processes[CITATION NEEDED: FORTES E REIS], Developer productivitys[CITATION NEEDED], Task Completions[CITATION NEEDED], Developer Attraction and Retentions[CITATION NEEDED].
 
TODO: Introduce: PGA[CITATION NEEDED:AULA 4 CSCW]


\section{Methods}

\subsection{Data Analysis}

\subsubsection{Source Code}

\subsubsection{Issues}

\subsubsection{Software Downloads}

\subsubsection{Message Exchange}

\subsection{Questionnaires}

\section{Results}

\section{Discussion}

\section{Conclusion}

\section{Acknowledgments}

Sample text: We thank all the volunteers, and all publications support
and staff, who wrote and provided helpful comments on previous
versions of this document. Authors 1, 2, and 3 gratefully acknowledge
the grant from NSF (\#1234--2012--ABC). \textit{This whole paragraph is
  just an example.}


% Balancing columns in a ref list is a bit of a pain because you
% either use a hack like flushend or balance, or manually insert
% a column break.  http://www.tex.ac.uk/cgi-bin/texfaq2html?label=balance
% multicols doesn't work because we're already in two-column mode,
% and flushend isn't awesome, so I choose balance.  See this
% for more info: http://cs.brown.edu/system/software/latex/doc/balance.pdf
%
% Note that in a perfect world balance wants to be in the first
% column of the last page.
%
% If balance doesn't work for you, you can remove that and
% hard-code a column break into the bbl file right before you
% submit:
%
% http://stackoverflow.com/questions/2149854/how-to-manually-equalize-columns-
% in-an-ieee-paper-if-using-bibtex
%
% Or, just remove \balance and give up on balancing the last page.
%
\balance{}

% REFERENCES FORMAT
% References must be the same font size as other body text.
\bibliographystyle{SIGCHI-Reference-Format}
\bibliography{sample}

\end{document}

%%% Local Variables:
%%% mode: latex
%%% TeX-master: t
%%% End:
